\documentclass{article}
\usepackage{amsmath}
\usepackage{amssymb}
\usepackage[makeroom]{cancel}

\usepackage{import}
\usepackage{pdfpages}
\usepackage{transparent}
\usepackage{xcolor}

\newcommand{\incfig}[2][1]{%
    \def\svgwidth{#1\columnwidth}
    \import{./figures/}{#2.pdf_tex}
}

\pdfsuppresswarningpagegroup=1

%\addtolength{\textheight}{+ .1\textheight}
\title{HW 4 Theoretical Part}
\author{Kevin Corcoran}
%\date{}
\begin{document}
\maketitle

\section{Problem 1}%
\label{sec:problem_1}

\subsection{Part 1}%
\label{sub:part_1}

\textbf{Carry out von Neumann stability analysis for the BTCS method} 

\vspace{15px}
\par Plug in $u_{i}^{n} = p^{n}e^{\sqrt{-1}\xi i\Delta x}$ to the BTCS method

\[
u_{i}^{n+1}=u_{i}^{n} + r \left(u_{i+1}^{n+1}-2u_{i}^{n+1}+u_{i-1}^{n+1}\right)
.\] 

\begin{align*}
  p^{n+1}e^{\sqrt{-1}\xi i\Delta x} &= p^{n}e^{\sqrt{-1}\xi i\Delta x}
  + r p^{n+1}e^{\sqrt{-1}\xi i\Delta x} \left(e^{\sqrt{-1}\xi\Delta x}
  - 2 + e^{-\sqrt{-1}\xi\Delta x} \right) \\
\end{align*}

\par Solving for $p$

\begin{align*}
  \left(e^{\sqrt{-1}\xi i\Delta x} -  r p^{n+1}e^{\sqrt{-1}\xi i\Delta x} \left(e^{\sqrt{-1}\xi\Delta x}
  - 2 + e^{-\sqrt{-1}\xi\Delta x}\right)\right) p^{n}p &= p^{n}e^{\sqrt{-1}\xi
i\Delta x} \\
\end{align*}

\begin{align*}
\implies p &= \frac{1}{1 - r \left(e^{\sqrt{-1}\xi \Delta x}
- 2 + e^{-\sqrt{-1}\xi \Delta x}\right)} \\
           &= \frac{1}{1 - 2r \left(\cos(\xi\Delta x) - 1\right) } \\
 &= \frac{1}{1 + 4r \left( \sin^2( \frac{\xi\Delta x}{2})\right) } 
\end{align*}


\vspace{15px}
\par We need $|p| \leq 1 + c\Delta t$. 
\vspace{15px}

Since $1 + 4r \left( \sin^2(
\frac{\xi\Delta x}{2})\right) \geq 1 \quad \forall\xi$

\[
\implies |p| = \left|\frac{1}{1 + 4r \left( \sin^2( \frac{\xi\Delta
x}{2})\right) } \right| \leq 1 + c\Delta t
.\] 

\vspace{15px}
\par so this method is unconditionally stable.

\section{Problem 2}%
\label{sec:problem_2}

\subsection{Part 1}%
\label{sub:part_1}

\textbf{Show $\lambda^{k} = 2(\cos(k\pi \Delta x) - 1)$ and $\omega^{k}
= \sin(k\pi i \Delta x) $are eigenvalues and eigenvectors of the matrix $(
\Delta x)^{2}A$.} 
\vspace{15px}

\par Consider the $i^{th}$ row of $A \omega^{k}$
\begin{align*}
  (A \omega^{k})_{i} &= \sin(k\pi(i-1)\Delta x) - 2\sin(k\pi i\Delta x)
  + \sin(k\pi (i+1)\Delta x) \\ 
                     &= \sin(k\pi i \Delta x - k\pi \Delta x) + \sin(k\pi
                     i \Delta x + k\pi\Delta x) - 2\sin(k\pi i \Delta x) \\
                     &= \sin(k\pi i \Delta x)\cos(k\pi \Delta x)
                     - \cancel{\cos(k\pi i \Delta x)\sin(k\pi \Delta x)}
                     + \sin(k\pi i \Delta x)\cos(k\pi \Delta x)
                     +\cancel{\cos(k\pi i \Delta x)\sin(k\pi \Delta x)}
                     - 2\sin(k\pi i \Delta x)\\
                     &=  \\
\end{align*}

This shows $2(\cos(k\pi \Delta x) - 1)$ is an eigenvalue and $\sin(k\pi
i \Delta x)$ is an eigenvector.


\subsection{Part 2}%
\label{sub:part_2}

Do the same for the vector $\cos(k\pi i \Delta x)$
\[
  \cos(k\pi(i-1)\Delta x) - 2\cos(k\pi i\Delta x) + \cos(k\pi(i+1) \Delta x)
.\] 

\subsection{Part 3}%
\label{sub:part_3}

We need zero boundary condition at $i=0$ and $i = N+1$.  $\cos(\dots i \dots)$ doesn't do that for us.


\end{document}
